%Classe de document
\documentclass[11pt,a4paper]{report}

%Francisation
\usepackage [utf8] {inputenc} 
\usepackage [LGR,T1] {fontenc} % Saisie fran??ais + grec
\usepackage [frenchb] {babel}


\usepackage[parfill]{parskip}
\usepackage{allfeta}
\usepackage{multicol}

\makeindex
\begin{document}
%--------------------------PAGE DE GARDE-------------------
%----------------------------------------------------------------------
\thispagestyle{empty}
\pagenumbering{roman}
\begin{center}
\begin{LARGE}
Caractères de la police FETA\end{LARGE}
\end{center}



%------------------------------------INTRODUCTION--------------------
%------------------------------------------------------------------------------
%--------------------------CHAPITRES------------ ----------------
%------------------------------------------------------------------------------

Guitar music is written with a treble clef \fetadeuxcentcinq{} rather than 
  a bass clef \fetadeuxcenttrois{}, which is correctly indicated by a small 8 
  under the treble clef \fetadeuxcentcinq{}.


\begin{table}[htdp]
\begin{center}
\begin{tabular}{|c|c|}
\fetacenttrenteetun & \fetacentsoixantedeux \\
%& \\
\fetadeuxcentun & \fetadeuxcentcinq \\
\end{tabular}
\end{center}
\caption{test}
\label{default}
\end{table}%



\end{document}